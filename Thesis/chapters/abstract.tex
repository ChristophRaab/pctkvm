% German Abstract
\begin{center}
 	\Large\textbf{Abstrakt}\\
\end{center}
In der Mustererkennung gilt die Klassifizierung, basierend auf überwachtem Lernen, als ein wichtiges Paradigma und findet Anwendungen in vielen Bereichen. Auf Grund von bereits kategorisierten Trainings Daten, können Algorithmen, die auf überwachtem Lernen basieren, ein Modell konstruieren. Mithilfe dieses Modells können vorhersagen erstellt werden. Die Qualität der Vorhersage ist jedoch schlecht, wenn die Trainingsdaten in diesem Bereich nicht ausreichen oder von einem anderen Bereich kommen und daher verschieden sind. Dies ist üblich realen Anwendungsfällen. Die Probabilistic Classification Vector Machine ist zwar in der Lage viele Probleme der momentanen Standardlösung zu beheben, kann jedoch die Unterschiede in den Bereichen nicht ausgleichen. Um das Problem des Wissenstransfers zu beheben, wurde die Probabilistic Classification Vector Machine durch das Transfer-Lernen erweitert. Dies basiert auf dem Transfer-Kernel-Ansatz. Außerdem wurden das daraus entstehende Problem der aufwändigen Parametersuche durch einen heuristischen Ansatz ersetzt. Die Leistungsunterschiede der Algorithmen wurden gegenübergestellt und zudem mit momentanen Transfer-Learning-Lösungen verglichen. Die Tests wurden auf, den für die Wissenschaft üblichen, Datensätzen und außerdem auf einem realen Anwendungsfall durchgeführt. Die Unterschiede wurden mit Hilfe der Kreuzvalidierung und dem Friedman Test festgestellt. Es konnte bewiesen werden, dass die vorgenommenen Änderungen die Leistung des Algorithmus signifikant verbessern. Generell kann gesagt werden, dass unter den gängigen Labortests die Qualität recht gut ist, jedoch konnte gezeigt werden, dass der Unterschied bei realen Anwendung nicht die Signifikanz der Labortests aufweist. Daher müssen weitere Überlegungen zur (verbesserten) Anwendungen von Transfer-Algorithmen in realen Fällen vorgenommen werden.
\newpage

% English Abstract
\begin{center}
	\Large\textbf{Abstract}\\
\end{center}
%Motivation
One prominent paradigm in the field of pattern recognition which has many applications in a broad range of disciplines is classification based on supervised learning. 
Algorithms, which are based on supervised learning creating their model based on pre-classified training data.
This model can predict future events and can classify new information.
%Problem
However, a problem where classification suffers from is that the domain of interest, where the prediction takes place, has no sufficient training data. 
Furthermore, if the only available training data is drawn from another domain, hence different, then it will result in a critical performance drop.
Domain differences are quite common in real-world scenarios.
The probabilistic classification vector machine is being able to ease some critical issues from the current state of the art solution but suffers from the transfer problem.
% Approach 
To tackle this issue, we will extend the probabilistic classification vector machine with transfer learning through a transfer kernel approach and solve the resulting problem of parameter determination through a heuristic estimation. 
% Results 
We have determined the performance differences of these two algorithms and compared it to other transfer learning approaches based on well-known datasets and a real-world dataset. 
The difference is determined via cross-validation and the Friedman test.
It can be shown that our improvements are resulting in a significant rise of performance, which makes it competitive to other transfer learning approaches.
%Conclusion
Summarizing, although the performance under lab conditions is pretty well, when it comes to real-world scenario, the benefits are lowered.
Therefore, future research must be done to make it more applicable to real-world problems.
