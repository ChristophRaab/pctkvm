\chapter{Conclusion}\label{ConChap}	
%TODO: Mehr ausformulieren
In the course of this work, we have given a general understanding of transfer learning with many partial sub-problems and applications.
Moreover, we gained knowledge on how probabilistic classification works and discussed advantages and disadvantages concerning the algorithms.
One of many key feature of this \acs{PCVM} is to deal with the uncertainty or latent state of underlying structure.\\
Furthermore, we described how one could use transfer learning to boost the performance of the probabilistic classification vector machines.
Additionally, we discussed standard methods to create a reliable study.
This statistics vary in their methodology and addressing different constitutions of datasets.
We have successfully created a new transfer learning algorithm and obtained many insights.\\
Especially the behaviour, the potential and limits of transfer learning methods
We gained knowledge of the efficiency of the different Gaussian kernels concerning the parameters.
It can be seen that the probabilistic classification vector machine can be greatly improved with the help of transfer learning and is finally prepared for difficult, especially transfer learning, tasks.
The result is the \acl{PCTKVM} which is comparable to many currently available transfer learning solutions.\\ 
Summarizing, in general transfer learning aims to improve the traditional supervised classification significantly.
It can be shown that by integrating it, the performance can be greatly improved.
This can be verified with several cross-domain datasets and statistical tests.\\
The state of current research shows a great amount of transfer learning methods.\cite[p. 33]{Weiss.2016}
All of them are solving the problem in their way with an individual set of requirements, which results naturally in different performances, concerning tasks or domains in general.
Based on different sub problems in transfer learning, the current state is that domain adaptations will lower the differences in marginal distributions. Furthermore, to address the problem of different conditional probability distributions the pseudo labelling technique will be applied. When it comes to different tasks, an inductive approach with a small amount of labelled target data is also sufficient to align the differences.
If one is faced with the problem that the feature spaces of two domains are not equal, e.\,g. different metrics, the use of heterogeneous transfer learning solutions may be helpful.
Moreover, heterogeneous transfer learning can be used to create a sentiment bridge between two domains to add information to a certain domain in order to improve the classification.
This results in a wide range of approaches which can be considered if one wants to integrate transfer learning in his application to meet his unique requirements and situations.\\
Moreover, traditional supervised machine learning algorithms are greatly challenged in real-world applications or difficult datasets.\cite{Pan.2010}
This challenge can be addressed with transfer learning to make machine learning in general more practical for real-world problems.
Especially when it the lab conditions are left behind. 
Our example real-world problem is the re-calibration of a myoelectric prosthesis, which classifies motion sequences in movements. 
This problem can be addressed with transfer learning to improve quality of the recognition rate, which can finally improve the quality of the patient's life.
The range of type from real-world problems is very broad.
Some examples are sentiment classification from very different types of text document, difficulty image classification, the detection of defect prone software.\cite{Weiss.2016}
However, more important the task of human activity recognition, cancer perdition or other healthcare related tasks.\cite{Burlina.2017}\cite{Kourou.2015}\\
The incorporating of machine learning in this research sections, not only our prosthesis case, has the potential to improve the health or life of patients greatly.
Although that transfer learning does a good job, there are still problems and potential left for the supervising learning task.
This issues should be addressed to improve the performance of transfer learning not only under lab conditions but also in the real world.
This can help to reach a new level of supervised classification performance in the future.
